
%%%%%%%%%%%%%%%%%%%%%%%%%
% 数据结构大作业 项目介绍
%%%%%%%%%%%%%%%%%%%%%%%%%

\section*{项目介绍}
我们的图书管理系统的名称叫做 \href{http://yrarbil.iok.la}{Yrarbil} 是 图书馆英文单词 \textbf{library} 的逆写。\footnotetext{我们喜爱开源项目与和开源相关的创业项目。}
\subsection*{团队介绍}
XDUDsTeam 开发团队由 西安电子科技大学软件学院1413014班的五名同学组成。他们依次是:
\paragraph{李约瀚} 李约瀚就读于 西安电子科技大学软件学院软件工程专业,目前是一名大二学生。
主要使用 Haskell C\# C++ 等语言进行开发。
\subsection*{项目介绍}
该项目是一个简单地图书馆图书管理系统。可用于小型图书馆的图书管理工作。

该项目是起于西安电子科技大学软件学院2015年数据结构课程大作业。目前遵循 BSD3 开原协议 \footnote{有计划更换开原协议。}。这个项目后端使用以 \href{https://www.haskell.org}{Haskell} 编写的框架 wrap 与 wai \footnote{其中主要是用到了一个名为 \href{https://www.yesodweb.com}{Yesod} 的框架}。同时使用著名的开源数据库
\href{http://www.postgresql.org}{PostgreSQL} ,作为存储数据的方式。
整个项目使用 \href{https://travis-ci.org}{Travis-ci} 持续集成服务,与 \href{https://www.docker.com/}{Docker} 容器服务。同时计划在DaoCloud \footnote{道客云,国内的Docker服务创业公司。}
\subsection*{项目信息}
这个项目是遵行 BSD3 开源协议开源的。托管于 GitHub 上的   \href{https://github.com/XDUDsTeam/}{Repo}。
我们计划将把稳定的编译后的二进制程序通过 Docker 镜像直接发布于
\href{https://hub.docker.com/}{Docker Hub}
\footnote{Docker Hub 可能会被GFW封锁。\tiny{至于原因,帝国主义反动势力在作祟。}如果你在XDU的校内网访问的话,可能也无法访问。上述GFW的制作者在XDU演讲时当场打脸。}
上。