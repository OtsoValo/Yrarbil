%%%%%%%%%%%%%%%%%%%%%%%%%%%%%%%%%%
% 数据结构大作业 Yrarbil API
%%%%%%%%%%%%%%%%%%%%%%%%%%%%%%%%%%

%%%%%%%%%%%%%%%%%%%%%%%%%%%%%%%%%%
%% Version 0.1.0.0-base
%%%%%%%%%%%%%%%%%%%%%%%%%%%%%%%%%%



%确定该文档
\def\apidPart{}
%%%%%%
%% 版本
%%%%%%
\def\apiver{0.1.0.0-base}

%%%%%%
% flag 设置
%%%%%%
\makeatletter
\def\@NoStyleChapter{} % 设置不适用章节格式
\def\@apiPart{} % 设置当前为 api 部分
\def\@APIVersionFlag{}
\def\@RequestArgColor{\relax}
\def\@UsingAppendix{}
\def\@FootnoteWithPage{}
\def\@lst@json@define{} %json
\makeatother

%使用 XeLaTeX 和 XeCJK 套装 编译
\documentclass[UTF8]{article}
\usepackage{xeCJK}


%%%%%%%%
% 数据结构大作业 报告作者 
%%%%%%%%


\author{李约瀚 \\ qinka@live.com \\ 14130140331 
    \and 褚欣 \\ m15949075919@163.com \\14130140356
    \and 戚瑶 \\ 631987611@qq.com \\14130140362
    \and 乔新文 \\ starsriver@outlook.com \\14130140393  
    \and 殷熔磾 \\ yinrongdi@163.com \\ 14050120069 }
%%%%%%%%%%%%%%%%%%%%%%
% 数据结构大作业 报告导言区
%%%%%%%%%%%%%%%%%%%%%%


\setCJKmainfont{WenQuanYi Zen Hei}
% 使用 color 宏包
\makeatletter
\ifdefined\@NoPagckageColor %! 如果定义了 ,不包含宏包color
\relax
\else
\usepackage{color}
\fi
\makeatother



% 对链接等的处理
\makeatletter
\ifdefined\@NoPackageHyperref
\relax
\else\usepackage[colorlinks,linkcolor=blue,anchorcolor=blue,citecolor=red,bookmarksnumbered]{hyperref}
\fi
\makeatother


% 使用 代码环境的宏包
\makeatletter
\ifdefined\@NoPackageListings
\relax
\else
\usepackage{listings}
\usepackage{xcolor}
\lstset{breaklines}
\lstset{basicstyle=\sffamily,keywordstyle=\bfseries,commentstyle=\rmfamily\itshape,escapechar='}
\lstset{flexiblecolumns}
\fi
\makeatother


%调整样式
%修改页眉与页脚
\usepackage{fancyhdr}
\pagestyle{fancy}



%章节的
\usepackage{titlesec}
%目录的
\usepackage{titletoc}



%修改Chapter的格式
\makeatletter
\ifdefined\@NoStyleChapter
\relax
\else\titleformat{\chapter}
    [display]
    {\centering\Huge\bfseries}
    {第\,\thechapter\,部分}
    {1em}
    {}
\fi
%修改Chapter的目录格式
\titlecontents{chapter}
[0pt]
{\addvspace{2pt}\filright}
{\contentspush{}}%\thecontentslabel\ }}
{a}
{\titlerule*[8pt]{.}\contentspage}
\makeatother

%定义 BibTeX 图标
\def\BibTeX{{\rm B\kern-.05em{\sc i\kern-.025em b}\kern-.08em
		T\kern-.1667em\lower.7ex\hbox{E}\kern-.125emX}}


%对于 api 版本的控制
\makeatletter
\ifx \apiver \@empty
\def\apiver{!!}
\fi
\ifdefined\@APIVersionFlag
\def\Color@now#1{\colorbox[rgb]{0.98,0.72,0.43}{#1}}
\def\Color@old#1{\colorbox[rgb]{0.79,0.79,0.58}{#1}}
\def\APIColor#1#2{
    \def\tmp@a{#1}
    \def\tmp@b{#2}
    \ifx \tmp@a \tmp@b
    \Color@new{#1}
    \else
    \Color@old{#1}
    \fi
    }
\def\apiversionn#1{\paragraph{\APIColor{\small#1}{\apiver}}}
\def\apiversion#1#2{\paragraph{{\colorbox[rgb]{0.98,0.72,0.43}{#1}} #2}\stepcounter{paragraph}}
\else
\def\apiversion#1{#1}
\def\apiversionn#1{#1}
\fi
\makeatother




%对于附录的设置
\makeatletter
\ifdefined\@UsingAppendix
\usepackage[titletoc]{appendix}
\renewcommand\appendixname{附录}
\renewcommand\appendixpagename{附录}
\appendixtitleon
\else
\relax
\fi
\makeatother

% 设置HTTP请求参数显示的背景色
\makeatletter
\ifdefined\@RequestArgColor
\def\reqargclr#1{\item[{\colorbox[rgb]{1.00,0.90,1.00}{#1}}]}
\fi
\makeatother


% 使得脚注标号随页码更新
\makeatletter

\ifdefined\@FootnoteWithPage
\usepackage{chngcntr}
\counterwithin{paragraph}{section}
\counterwithin{paragraph}{subsection}
\counterwithin{paragraph}{subsubsection}
\counterwithin{footnote}{paragraph}
\renewcommand\thefootnote{\theparagraph-\alph{footnote}}
\fi
\makeatother


% 定义 JSON
\makeatletter
\ifdefined\@NoPackageListings
\relax
\else
\ifdefined\@lst@json@define
\lstdefinelanguage{JSON}
{
    keywords={true,false,null}
}
\fi
\lstdefinelanguage{JavaScript}%
{
    morekeywords={true,false,null},%
    alsoletter={:},
    moredelim=[s][{\color[rgb]{0.67,0.00,0.67}}]{"}{"},
    moredelim=[s][{\color[rgb]{0.36,0.67,0.78}}]{:"}{"},%
    identifierstyle=\color{blue},
    emph={:}, emphstyle=\color{red}
}[keywords,comments,strings]%
\fi
\makeatother


\title{Yrarbil's API Document \thanks{version : \apiver}}

%%%%
%% API 常用
%%%%
\def\safe{[\textit{safe}]}
\def\GET{\textbf{GET}}
\def\POST{\textbf{POST}}
\def \itemPOST{\item[访问方式] \POST}
\def \itemGET{\item[访问方式] \GET}
\def\bfJSON{\textbf{JSON}}
\def\viaurl#1{\item[访问URL] #1}
\def\viareq#1{\item[访问方式] #1}
\def\rtdata{\item[返回数据]}





\begin{document}
    \maketitle
    \newpage
    
%%%%%%%%%%%%%%%%%%%%%%%%%
% 数据结构大作业 项目介绍
%%%%%%%%%%%%%%%%%%%%%%%%%

\section*{项目介绍}
\subsection*{团队介绍}
XDUDsTeam 开发团队由 西安电子科技大学软件学院1413014班的五名同学组成。他们依次是:
\paragraph{李约瀚} 李约瀚就读于 西安电子科技大学软件学院软件工程专业,目前是一名大二学生。
主要使用 Haskell C\# C++ 等语言进行开发。
\subsection*{项目介绍}
该项目是一个简单地图书馆图书管理系统。可用于小型图书馆的图书管理工作。

该项目是起于西安电子科技大学软件学院2015年数据结构课程大作业。目前遵循 BSD3 开原协议 \footnote{有计划更换开原协议。}。这个项目后端使用以 Haskell 编写的框架 wrap 与 wai \footnote{其中主要是用到了一个名为 \href{https://www.yesodweb.com}{Yesod} 的框架}。同时使用著名的开源数据库
\href{http://www.postgresql.org}{PostgreSQL} ,作为存储数据的方式。
\subsection*{项目信息}
这个项目是遵行 BSD3 开源协议开源的。托管于 GitHub 上的   \href{https://github.com/XDUDsTeam/}{Repo}
    \pdfbookmark[1]{目录}{anchor}
    \tableofcontents
    \newpage


    \section{API 改动说明}

    \subsection{API 0.0.0.0-dbtag}
    该 API 标志的,首先,意味着这个数据库可查询其中支持的各个 API 版本。其次,意味着这是一个标准
      \footnote{按照这个API 文档构件的后端。并且从理论上,非配套的前后端也可以交互。}
    后端。最后,一般前端是无法直接访问数据库的,然而后端是要访问数据库的。

    \subsection{API 0.0.0.0-betag}
    该 API 标志着可以通过特定方式查询后端与数据库支持的 API 的版本。这也是一个标准后端的应有的。

    \subsection{DEL 0.0.0.0-tag,0.0.0.0-B,0.0.1.0-base}
    重新 整理API。

    \section{API说明}
    本 API 文档是约束前后端的行为的文档。规定了前后端之间的行为规范。但值得注意的是,前后端不一定是满足这个文档的。当前的 API 版本是 \apiver 。
    \subsection{那些部分会遵循api}
    原则上来说,整个项目均会按照此 api 文档构建。但主要的是在数据库,后端,与前端部分支持。













    \section{数据库API}
    \apiversion{0.0.0.0-dstag}{版本检查}
    可查询表 versions 来确定 所需要的 api 对应的版本号是否支持。
    \begin{lstlisting}[language=SQL]
SELECT * FROM versions WHERE NOT vereq(ver,yourver);
    \end{lstlisting}
    其中 yourver 是指定的版本标号。




















    \section{后端API}
    后端 API 主要以 HTTP 请求为主。


    \subsection{未分类API}

    \apiversion{0.1.0.0-base}{访问“主页”}
    访问主页
    \footnote{http://hostname:port/}
    可以获取一些信息。
    \begin{description}
        \viaurl http://hostname:port/
        \viareq \GET
        \rtdata 标准 HTML 代码。
    \end{description}

    \apiversion{0.0.1.0-base}{获得 api 版本}
    获得当前 api 版本支持。
    \begin{description}
		\item[访问 URL] http://hostname:port/version
		\itemGET
		\reqargclr{NoJSON 选项}
		$$nojson=[true|false]$$
		如果是 $true$ 那么则不返回 JSON 数据,默认是 $false$。
	\end{description}









    \subsection{用户认证API}











    \apiversion{0.1.0.0-base}{管理员登陆}
    \begin{description}

        \viaurl http://hostname:port/b8ab91a6
        \footnote{
            \textbf{login} 英文单词 md5值是$b8abe8f5a587a4c69685b17a954f8a3f$,
            SHA256值是
            $6c1420b8f3be7a61357ecc82a1c391dffe438930c103466b1660625e42ac91a6$。
        }
        \safe
        \footnote{\textit{\textbf{safe}} 是指安全添加的内容,例如可以是主机ip的sha256值。一个URL中的“safe”不一定相同。后同}/fc8252b7
        \footnote{
            \textbf{admin} 英文单词的
            md5值是 $456b7016a916a4b178dd72b947c152b7 $,
            SHA256 值是
            $fc8252c8dc55839967c58b9ad755a59b61b67c13227ddae4bd3f78a38bf394f7$。
        }
        \safe

        \viareq \POST

        \reqargclr{用户ID}\label{par:id}
        \footnote{这个颜色的背景,表示HTTP请求参数,后同。}
        $$uid=[id]$$
        其中id为用户登录的id,不强制为数字。

        \reqargclr{密码}\label{par:key}
         $$key=[md5\, of\, password][safe]$$
         其中第一部分是密码的md5值。

        \rtdata 请求返回的数据是 \bfJSON 数据。具体返回内容有:
        特定的临时ID密钥,用于简化之后操作的认证;
        时限,长度为小时,限定使用实现\footnote{可选,可无}。
         \\ 可能的返回数据:
        \begin{lstlisting}[language=JavaScript]
{
   "tidk":"************************",
   "time":24
}
        \end{lstlisting}


    \end{description}








    \apiversion{0.1.0.0-base}{管理员退出}
    \begin{description}
        \viaurl http://hostname:port/a050ba73
        \footnote{
            单词 \textbf{logout} 的
            MD5值是$a05080965b04f4d190bb94b552bea56c $,
            SHA256 值是
            $60b2a0ba73317dce9502417ce5bdb75d0a16067848647b9a5f5f4e040397b20f$。
        }
        \safe/fc8252b7\safe]

        \viareq \POST

        \reqargclr{用户ID}\label{par:logout:id}
        \footnote{这个请求与临时ID密钥为二选一。}
        $$uid=[id]$$
        其中id为用户登陆时的id,不强制为数字。

        \reqargclr{临时ID密钥}\label{par:logout:tidk}
        \footnote{用户ID,可使该用户对应的所有临时ID密钥下线。}
        $$tidk=[key]$$
        其中key是临时ID密钥。

        \rtdata 返回数据将是 \bfJSON 数据。其中会包含成功与否的信息。
        \\可能返回的信息:
        \begin{lstlisting}[language=JavaScript]
{
    "status":"success",
    "logoutcount":2
}
        \end{lstlisting}\label{par:logout:example}
        其中 \textbf{logoutcount} 是退出用户数量。
    \end{description}






    \apiversion{0.1.0.0-base}{读者登录}
		用于读者的登录。
		\begin{description}

		\viaurl http://hostname:port/5ece85ba
        \footnote{单词 \textbf{reader} 的
            MD5 值是
            $5ece7ae3f6950b744840b1d4c42e3a8e$,
            SHA256 值是
            $cd546fe85ba55d958a7fd1b2733f6f8f1ba5e4e099ddf0aad80b6d770b2eed9e$。
            }
        \safe/fc8252b7\safe

        \viareq \POST

        \reqargclr{用户ID} 参见 \ref{par:id}。
        \reqargclr{密码} 参见 \ref{par:key}。
        \rtdata 返回的数据将是 \bfJSON 数据。返回的内容中包括临时ID密钥,与有效时间,以分钟计。
        \\可能返回的数据:
        \begin{lstlisting}[language=JavaScript]
{
    "tidk":"****************",
    "time":120
}
        \end{lstlisting}
		\end{description}





        \apiversion{0.1.0.0-base}{读者退出}
        \begin{description}
            \viaurl http://hostname:port/5ece85ba\safe/fc8252b7\safe

            \viareq \POST

            \reqargclr{用户ID}
            参见 \ref{par:logout:id}。

            \reqargclr{临时ID密钥}
            参见 \ref{par:logout:tidk}。


            \rtdata 返回数据将是 \bfJSON 数据。其中会包含成功与否的信息。
            \\可能返回的信息:
            \begin{lstlisting}[language=JavaScript]
{
    "status":"failure"
}
            \end{lstlisting}
            另外一个例子请参见 \ref{par:logout:example}。
        \end{description}


        \subsection{图书借阅API}
        \footnote{借阅,归还,续借}
        \apiversion{0.1.0.0-base}{图书借阅}
        \apiversion{0.1.0.0-base}{图书归还}
        \apiversion{0.1.0.0-base}{图书续借}
        \apiversion{0.1.0.0-base}{超期处理}
        \apiversion{0.1.0.0-base}{处罚处理}%缴纳罚款
        \subsection{图书预约API}
        \apiversion{0.1.0.0-base}{图书预约}
        \apiversion{0.1.0.0-base}{图书预约取消}
        \subsection{图书信息查询}
        \apiversion{0.1.0.0-base}{图书状态查询}
        \apiversion{0.1.0.0-base}{图书CIP信息查询}
        \apiversion{0.1.0.0-base}{图书其他信息查询}
        \subsection{图书检索}
        \apiversion{0.1.0.0-base}{根据ISBN匹配}
        \apiversion{0.1.0.0-base}{密码}
        \subsection{图书录入与销毁}
        \apiversion{0.1.0.0-base}{图书录入}
        \apiversion{0.1.0.0-base}{图书销毁}





        \subsection{图书的生命周期部分}
        图书的生命周期包括被查询,被预约,借阅,续借,归还,和采购与销毁。






        \apiversion{0.0.1.0-base}{查询图书的 CIP 数据}
        通过 ISBN查询,返回图书的信息。
        \begin{description}

        	\viaurl http://hostname:port/
        	423472651c
        	\footnote{单词 \textbf{book}的
        		MD5 值是 $4f98663772651c870e911982e991d0d9$,
        		SHA256 值是
                $3c4dbff9eeda9410a0b9eb423472981db8997d666388b4cd2424700c3974d64b$
                。
        		}
	        \safe/9c4f7bfef
	        \footnote{
	        	单词 \textbf{search} 的
	        	MD5 值是 $9c4fc45b8e6ad4bf08f959548f52099e$,
	        	SHA256 值是
                $f1f3528bad895d7e9db4544081b531c2126781843aebafa6cbea7c2f9358cf85$
                。
	        	单词 \textbf{ISBN} 的
	        	MD5 值是 $f329188d04f7bd852894eed846386423$,
	        	SHA256 值是
                $3cb720e9f89bfef38a513b1ee0c47cc0ae75c3353612c3466761df19dc945f29$
                 。
	        	}
	        \safe


	        \viareq \GET


	        \reqargclr{ISBN码}
	        $$isbn=[ISBN]$$
	        其中的 ISBN 是书的 ISBN 码。


	        \reqargclr{返回数据形式}
	        $$type=[json|html]$$
	        返回的数据格式类型是 JSON 数据还是 HTML 文件。

	        \rtdata 请求返回的可能是数据,也可能是 html。
	        可能返回的数据之一:
\begin{lstlisting}[language=JavaScript]
{
	"name":"the name of book",
	"author":"who write it!",
	"publishLocation":"Xi`an,China",
	"pressHouse":"ourPress",
	"publishDate":"1970-01-01",
	"ISBN":"ISBN978-7-302-22446-4",
	"location":"TP312"
}
\end{lstlisting}
	       或者会拿到 HTMl 代码:
\begin{lstlisting}[language=HTML]
<HTML>
	<body>
	<p> the name of book</p>
	<p> 'who write it' write this book </p>
	<p> 'our press' publish this book in Xi`an , China </p>
	<p> At Jan. 1st, 1970. </p>
	<p> ISBN is 978-7-302-22446-4</p>
	<p> You can find these books at TP312.</p>
	</body>
</HTML>
\end{lstlisting}
        \end{description}




        \apiversion{0.0.1.0-base}{x}






    \newpage
    \begin{appendices}
    \section{参考文档}
    \section{文档网址}
    todo
    \end{appendices}

\end{document}
