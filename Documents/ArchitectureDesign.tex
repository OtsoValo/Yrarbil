%%%%%%%%%%%%%%%%%%%%%%%%%%%%%%%%%
% 数据结构大作业 架构设计报告
%%%%%%%%%%%%%%%%%%%%%%%%%%%%%%%%%%

%使用 XeLaTeX 和 XeCJK 套装 编译
\documentclass[UTF8]{report}
\usepackage{xeCJK}

%flag 设置
\makeatletter
\def\@ADDoc{}
\def\@UsingAppendix{}
\makeatother

%%%%%%%%%%%%%%%%%%%%%%
% 数据结构大作业 报告导言区
%%%%%%%%%%%%%%%%%%%%%%


\setCJKmainfont{WenQuanYi Zen Hei}
% 使用 color 宏包
\makeatletter
\ifdefined\@NoPagckageColor %! 如果定义了 ,不包含宏包color
\relax
\else
\usepackage{color}
\fi
\makeatother



% 对链接等的处理
\makeatletter
\ifdefined\@NoPackageHyperref
\relax
\else\usepackage[colorlinks,linkcolor=blue,anchorcolor=blue,citecolor=red,bookmarksnumbered]{hyperref}
\fi
\makeatother


% 使用 代码环境的宏包
\makeatletter
\ifdefined\@NoPackageListings
\relax
\else
\usepackage{listings}
\usepackage{xcolor}
\lstset{breaklines}
\lstset{basicstyle=\sffamily,keywordstyle=\bfseries,commentstyle=\rmfamily\itshape,escapechar='}
\lstset{flexiblecolumns}
\fi
\makeatother


%调整样式
%修改页眉与页脚
\usepackage{fancyhdr}
\pagestyle{fancy}



%章节的
\usepackage{titlesec}
%目录的
\usepackage{titletoc}



%修改Chapter的格式
\makeatletter
\ifdefined\@NoStyleChapter
\relax
\else\titleformat{\chapter}
    [display]
    {\centering\Huge\bfseries}
    {第\,\thechapter\,部分}
    {1em}
    {}
\fi
%修改Chapter的目录格式
\titlecontents{chapter}
[0pt]
{\addvspace{2pt}\filright}
{\contentspush{}}%\thecontentslabel\ }}
{a}
{\titlerule*[8pt]{.}\contentspage}
\makeatother

%定义 BibTeX 图标
\def\BibTeX{{\rm B\kern-.05em{\sc i\kern-.025em b}\kern-.08em
		T\kern-.1667em\lower.7ex\hbox{E}\kern-.125emX}}


%对于 api 版本的控制
\makeatletter
\ifx \apiver \@empty
\def\apiver{!!}
\fi
\ifdefined\@APIVersionFlag
\def\Color@now#1{\colorbox[rgb]{0.98,0.72,0.43}{#1}}
\def\Color@old#1{\colorbox[rgb]{0.79,0.79,0.58}{#1}}
\def\APIColor#1#2{
    \def\tmp@a{#1}
    \def\tmp@b{#2}
    \ifx \tmp@a \tmp@b
    \Color@new{#1}
    \else
    \Color@old{#1}
    \fi
    }
\def\apiversionn#1{\paragraph{\APIColor{\small#1}{\apiver}}}
\def\apiversion#1#2{\paragraph{{\colorbox[rgb]{0.98,0.72,0.43}{#1}} #2}\stepcounter{paragraph}}
\else
\def\apiversion#1{#1}
\def\apiversionn#1{#1}
\fi
\makeatother




%对于附录的设置
\makeatletter
\ifdefined\@UsingAppendix
\usepackage[titletoc]{appendix}
\renewcommand\appendixname{附录}
\renewcommand\appendixpagename{附录}
\appendixtitleon
\else
\relax
\fi
\makeatother

% 设置HTTP请求参数显示的背景色
\makeatletter
\ifdefined\@RequestArgColor
\def\reqargclr#1{\item[{\colorbox[rgb]{1.00,0.90,1.00}{#1}}]}
\fi
\makeatother


% 使得脚注标号随页码更新
\makeatletter

\ifdefined\@FootnoteWithPage
\usepackage{chngcntr}
\counterwithin{paragraph}{section}
\counterwithin{paragraph}{subsection}
\counterwithin{paragraph}{subsubsection}
\counterwithin{footnote}{paragraph}
\renewcommand\thefootnote{\theparagraph-\alph{footnote}}
\fi
\makeatother


% 定义 JSON
\makeatletter
\ifdefined\@NoPackageListings
\relax
\else
\ifdefined\@lst@json@define
\lstdefinelanguage{JSON}
{
    keywords={true,false,null}
}
\fi
\lstdefinelanguage{JavaScript}%
{
    morekeywords={true,false,null},%
    alsoletter={:},
    moredelim=[s][{\color[rgb]{0.67,0.00,0.67}}]{"}{"},
    moredelim=[s][{\color[rgb]{0.36,0.67,0.78}}]{:"}{"},%
    identifierstyle=\color{blue},
    emph={:}, emphstyle=\color{red}
}[keywords,comments,strings]%
\fi
\makeatother

%%%%%%%%
% 数据结构大作业 报告作者 
%%%%%%%%


\author{李约瀚 \\ qinka@live.com \\ 14130140331 
    \and 褚欣 \\ m15949075919@163.com \\14130140356
    \and 戚瑶 \\ 631987611@qq.com \\14130140362
    \and 乔新文 \\ starsriver@outlook.com \\14130140393  
    \and 殷熔磾 \\ yinrongdi@163.com \\ 14050120069 }

\title{架构设计报告}
\begin{document}
    \maketitle
    % license.tex
\section*{LICENSE}\label{license}
\# BSD3 License

\# we try to change this license to yaml formal

\def\qqquad{\quad \qquad}
Copyright (c) 2015, XDUDsTeam \\
  \hspace*{4em} MEMBER\\
  \hspace*{6em} - name: Qinka\\
  \hspace*{6.75em}email: qinka@live.com\\
  \hspace*{6em} - name: qiyuyi\\
  \hspace*{6.75em}email: 631987611@qq.com\\
  \hspace*{6em} - name: starsriver\\
  \hspace*{6.75em}email: starsriver@outlook.com\\
  \hspace*{6em} - name: woyijkl1\\
  \hspace*{6.75em}email: yinrongdi@163.com\\
  \hspace*{6em} - name: daisycx \\
  \hspace*{6.75em}email: 729227860@qq.com
\vspace{0.5em}

All rights reserved.
\vspace*{0.5em}

Redistribution and use in source and binary forms, with or without
modification, are permitted provided that the following conditions are met:
\vspace{.5em}

- Redistributions of source code must retain the above copyright notice, this
  list of conditions and the following disclaimer.
\par \vspace{0.25em}
- Redistributions in binary form must reproduce the above copyright notice,
  this list of conditions and the following disclaimer in the documentation
  and/or other materials provided with the distribution.
\par \vspace{0.25em}
- Neither the name of Yrarbil nor the names of its
  contributors may be used to endorse or promote products derived from
  this software without specific prior written permission.

\vfill
\textbf{THIS SOFTWARE IS PROVIDED BY THE COPYRIGHT HOLDERS AND CONTRIBUTORS "AS IS"
AND ANY EXPRESS OR IMPLIED WARRANTIES, INCLUDING, BUT NOT LIMITED TO, THE
IMPLIED WARRANTIES OF MERCHANTABILITY AND FITNESS FOR A PARTICULAR PURPOSE ARE
DISCLAIMED. IN NO EVENT SHALL THE COPYRIGHT HOLDER OR CONTRIBUTORS BE LIABLE
FOR ANY DIRECT, INDIRECT, INCIDENTAL, SPECIAL, EXEMPLARY, OR CONSEQUENTIAL
DAMAGES (INCLUDING, BUT NOT LIMITED TO, PROCUREMENT OF SUBSTITUTE GOODS OR
SERVICES; LOSS OF USE, DATA, OR PROFITS; OR BUSINESS INTERRUPTION) HOWEVER
CAUSED AND ON ANY THEORY OF LIABILITY, WHETHER IN CONTRACT, STRICT LIABILITY,
OR TORT (INCLUDING NEGLIGENCE OR OTHERWISE) ARISING IN ANY WAY OUT OF THE USE
OF THIS SOFTWARE, EVEN IF ADVISED OF THE POSSIBILITY OF SUCH DAMAGE.}
\newpage
    
%%%%%%%%%%%%%%%%%%%%%%%%%
% 数据结构大作业 项目介绍
%%%%%%%%%%%%%%%%%%%%%%%%%

\section*{项目介绍}
\subsection*{团队介绍}
XDUDsTeam 开发团队由 西安电子科技大学软件学院1413014班的五名同学组成。他们依次是:
\paragraph{李约瀚} 李约瀚就读于 西安电子科技大学软件学院软件工程专业,目前是一名大二学生。
主要使用 Haskell C\# C++ 等语言进行开发。
\subsection*{项目介绍}
该项目是一个简单地图书馆图书管理系统。可用于小型图书馆的图书管理工作。

该项目是起于西安电子科技大学软件学院2015年数据结构课程大作业。目前遵循 BSD3 开原协议 \footnote{有计划更换开原协议。}。这个项目后端使用以 Haskell 编写的框架 wrap 与 wai \footnote{其中主要是用到了一个名为 \href{https://www.yesodweb.com}{Yesod} 的框架}。同时使用著名的开源数据库
\href{http://www.postgresql.org}{PostgreSQL} ,作为存储数据的方式。
\subsection*{项目信息}
这个项目是遵行 BSD3 开源协议开源的。托管于 GitHub 上的   \href{https://github.com/XDUDsTeam/}{Repo}
    \pdfbookmark[1]{目录}{anchor}
    \tableofcontents
    \part{前端架构}
    前端负责域用户的交互等行为。
    \chapter{桌面应用部分}
    该部分为图书管理员处理图书的各项工作用的客户端。
    相对应的设置,请保留到本地文件中。
    \section{UI/交互部分}
    \subsection{管理员登陆部分}
    管理员使用用户名与密码
    \footnote{改密码请遵寻UNIX类密码}
    登入管理系统的界面。
    \subsection{管理员退出部分}
    管理员确认退出的界面。
    \subsection{图书借阅部分}
    在该部分,图书管理员先输入读者条码,然后再输入读者所借阅的图书的条码。
    当一个读者借阅完之后,读者条码那一部分需要复位。
    \subsection{图书归还部分}
    这个步骤无需借阅着的条码。直接输入读者的条码即可。
    如果有超期\footnote{应该有系统判断是否超期。}、污损\footnote{该部分应该有图书管理员负责。}的情况
    \subsection{图书续借部分}
    直接输入条码即可,然而续借次数有限制。
    \subsection{借阅处罚部分}
    对于各种原因的罚款,进行处理。记录。
    \subsection{图书录入部分}
    将一本书录入到系统中。
    \subsection{图书处理部分}
    将由于各种原因的出书处理,记录在案,同时图书销毁。
    \section{数据处理部分}
    负责各类数据处理
    \subsection{数据请求}
    将UI部分要求的数据请求进行处理,然后请求。
	\begin{description}
	\item[登录] 将UI传入的数据,制作成所需访问的URL与对应参数,同时生成HTTP
		请求。
    \item[退出] 将 UI 传入的退出的数据,拼接成 HTTP 请求。
    \item[图书管理] 将 图书处理的数据,按照API 文档,组合成HTTP请求之后,传输。
	\end{description}
    \subsection{数据解析}
    将后端返回的数据处理之后,发回UI部分。
	\begin{description}
	\item[登录] 返回的是JSON数据,要求将JSON解析。JSON数据要么为具体内容
		\footnote{具体数据格式,详见API文档。}
		,要么数错误信息。
	\item[登出]
	\end{description}
    \subsection{数据处理}
    按照UI的需求,处理数据。
    \chapter{网页交互部分}
    基于网页的技术。
    \section{访客查询部分}
    查询现有图书信息,以及图书是否在架信息。
    \subsection{图书信息查询}
    图书的CIP信息
    \footnote{依据作者,图书题目,ISBN,,等信息查询。}
    ,图书的引索信息。提供图书的标准的 \BibTeX 数据。
    \subsection{图书状态查询}
    查询图书的在架状态,预约状态,借阅的最长期限。
    \section{读者查询部分}
    主要是指,查询并且操作。
    \subsection{预约}
    对于某一本图书的预约,可以完成的是对在架与借阅中的图书的预约。同时预约期间不可以被其他人借阅,而有一定期限。
    \subsection{续借}
    对于已经借阅的图书,进行续借,续借的次数有限制。
    \part{后端架构}
    \setcounter{chapter}{0}
    \chapter{数据库部分}
    直接参照 API 文档 与 后端参考文档。
    \\ 附 SQL 初始化文件,需要使用 psql 执行。
    \begin{lstlisting}[language=SQL]
-- 这是 Yrarbil 后端数据库初始化文件
-- 用于 PostgreSQL 的


------------------------------------
-- API
------------------------------------
-- 0.0.0.0-tag
-- 0.0.0.0-B
-- 0.1.0.0-base
------------------------------------


------------------------------------
-- FUNCTIONs DEFINEs
------------------------------------

------------------------------------
-- 删除已存在的表的
CREATE OR REPLACE FUNCTION
drop_all_table(user_name IN VARCHAR,schema_name IN VARCHAR)
RETURNS VOID
AS $$
DECLARE statements CURSOR FOR
SELECT tablename FROM pg_tables
WHERE tableowner = user_name AND
schemaname = schema_name;
BEGIN
FOR stmt IN statements LOOP
EXECUTE 'DROP TABLE ' || quote_ident(stmt.tablename) || ' CASCADE;';
END LOOP;
END;
$$ LANGUAGE plpgsql;

------------------------------------
-- 大小流水号的比较
CREATE OR REPLACE FUNCTION
func_sernum_eq(b TEXT,s INT,d DATE)
RETURNS BOOLEAN
AS $$
BEGIN
RETURN (b=s||'@'||d);
END;
$$ LANGUAGE plpgsql;



------------------------------------

SELECT drop_all_table('qinka','public');

------------------------------------
-- TABLES
------------------------------------

------------------------------------
-- 创建 版本号记录表
CREATE TABLE table_version
(
main	INT  NOT NULL,
snd		INT  ,
trd		INT  ,
fix		INT	 ,
tag		CHAR(40)
);

-- 添加数据
INSERT INTO table_version
VALUES(
-- 见 API
0,0,0,0,
'tag'
);
INSERT INTO table_version
VALUES(
-- 见 API
0,0,0,0,
'B'
);
INSERT INTO table_version
VALUES(
-- 见 API
0,1,0,0,
'base'
);


-------------------------------------
--  图书管理数据
-------------------------------------
-- 创建 图书记录表
CREATE TABLE table_bookinfo
(
isbn			BIGINT NOT NULL PRIMARY KEY,
bookname		CHAR(128),
author			CHAR(64),
publish_local	TEXT,
publish_house   CHAR(64),
publish_date	DATE,
library_local	TEXT,
library_index	CHAR(40)
);

--------------------------------------
-- 创建 图书实体记录
CREATE TABLE table_bookitem
(
barcode			BIGINT NOT NULL PRIMARY KEY,
isbn			BIGINT NOT NULL,
on_shelf		BOOLEAN NOT NULL,
is_there		BOOLEAN NOT NULL,
latest_opt_id	CHAR(64),
bought_price	INT,
bought_date 	DATE,
);
--------------------------------------
-- 创建 图书操作记录 入库(购买)
CREATE TABLE table_bookopt_in
(
big_serial_number			TEXT NOT NULL PRIMARY KEY,
bought_price				INT NOT NULL,
isbn						BIGINT NOT NULL,
barcode						BIGINT NOT NULL,
note						TEXT
);

--------------------------------------
-- 创建 图书操作记录 出库(销毁)
CREATE TABLE table_bookopt_out
(
big_serial_number	TEXT NOT NULL PRIMARY KEY,
reason				TEXT NOT NULL,
note				TEXT
);


--------------------------------------
-- 创建 图书操作记录
CREATE TABLE table_bookopt_main
(
big_serial_number			TEXT NOT NULL PRIMARY KEY,
reader_barcode				TEXT NOT NULL,
book_barcode				BIGINT NOT NULL,
times								INT NOT NULL,
return_date					DATE,
is_return						BOOLEAN NOT NULL
);

--------------------------------------
-- 创建 图书预约记录
CREATE TABLE table_bookopt_appointment
(
big_serial_number			TEXT NOT NULL PRIMARY KEY,
reader_barcode				TEXT NOT NULL,
book_barcode				BIGINT NOT NULL,
appointment_date			DATE NOT NULL
);

--------------------------------------
-- 创建 处罚记录
CREATE TABLE table_punish
(
big_serial_number			TEXT NOT NULL PRIMARY KEY,
reader_barcode				TEXT NOT NULL,
cash						MONEY NOT NULL,
reason						TEXT NOT NULL
);

--------------------------------------
-- 创建 读者数据
CREATE TABLE table_reader
(
barcode					TEXT NOT NULL PRIMARY KEY, --已修改
reader_name				TEXT NOT NULL,
idcard_type				CHAR(40),
idcard_id				TEXT,
debt					MONEY,
enter_password		TEXT NOT NULL DEFAULT '111111'
);




--------------------------------------
-- 创建 操作数据记录
CREATE TABLE table_opt
(
small_serial_number			INT 		NOT NULL,
opt_date					DATE 		NOT NULL,
opt_usr_type				SMALLINT	NOT NULL,
opt_usr_id					TEXT		NOT NULL
);


--------------------------------------
-- 创建 系统管理员数据
CREATE TABLE table_adminer
(
admin_id			INT NOT NULL PRIMARY KEY,
admin_passwd	TEXT NOT NULL DEFAULT '222222'
);
--------------------------------------
-- 临时 ID 秘钥 表
CREATE TABLE table_tmpidkey
(
tmpid TEXT NOT NULL PRIMARY KEY,
timeend TIMESTAMP WITH TIME ZONE NOT NULL,
id TEXT NOT NULL
);
    \end{lstlisting}
    \chapter{请求处理部分}
    直接参照 API 文档 与 后端参考文档。
    \newpage
    \begin{appendices}
    % 参考文档链接
\def\ApiDocPDF{\href{https://github.com/XDUDsTeam/YrarbilRelease/raw/master/APIDoc.pdf}{API Document}}
\def\ArchitectureDesignPDF{\href{https://github.com/XDUDsTeam/YrarbilRelease/raw/master/ArchitectureDesign.pdf}{Architecture Design Document}}
\def\BackEndPdf{\href{https://github.com/XDUDsTeam/YrarbilRelease/raw/master/BackEnd.pdf}{Backend Document}} 
\def\DemandAnalysisPdf{\href{https://github.com/XDUDsTeam/YrarbilRelease/raw/master/DemandAnalysis.pdf}{Demand Analysis Docment}}
\def\TestReportPDF{\href{https://github.com/XDUDsTeam/YrarbilRelease/raw/master/TestReport.pdf}{Text Report}}

\makeatletter
\ifdefined \@NoStyleChapter
\section{可参考文档}
\else
\chapter{可参考文档}
\fi
\makeatother


Github \& Travis-CI 版
\begin{itemize}
    \makeatletter
    \ifdefined \@APIDoc
    \relax
    \else
    \item \ApiDocPDF
    \fi
    
    \ifdefined \@ADDoc
    \relax
    \else
    \item \ArchitectureDesignPDF
    \fi
    
    \ifdefined \@BackendDoc
    \relax
    \else
    \item \BackEndPdf
    \fi
    
    \ifdefined \@DADoc
    \relax
    \else
    \item \DemandAnalysisPdf
    \fi
    
    \ifdefined \@TRDoc
    \relax
    \else
    \item \TestReportPDF
    \fi
    \makeatother
\end{itemize}
    
    \end{appendices}
\end{document}
