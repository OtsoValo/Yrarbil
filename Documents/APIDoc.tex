%%%%%%%%%%%%%%%%%%%%%%%%%%%%%%%%%%
% 数据结构大作业 Yrarbil API
%%%%%%%%%%%%%%%%%%%%%%%%%%%%%%%%%%

%%%%%%%%%%%%%%%%%%%%%%%%%%%%%%%%%%
%% Version 0.1.0.0-base
%%%%%%%%%%%%%%%%%%%%%%%%%%%%%%%%%%



%确定该文档
\def\apidPart{}
%%%%%%
%% 版本
%%%%%%
\def\apiver{0.1.0.0-base}

%%%%%%
% flag 设置
%%%%%%
\makeatletter
\def\@NoStyleChapter{} % 设置不适用章节格式
\def\@apiPart{} % 设置当前为 api 部分
\def\@APIVersionFlag{}
\def\@RequestArgColor{\relax}
\def\@UsingAppendix{}
\def\@FootnoteWithPage{}
\def\@lst@json@define{} %json
\makeatother

%使用 XeLaTeX 和 XeCJK 套装 编译
\documentclass[UTF8]{article}
\usepackage{xeCJK}


%%%%%%%%
% 数据结构大作业 报告作者 
%%%%%%%%


\author{李约瀚 \\ qinka@live.com \\ 14130140331 
    \and 褚欣 \\ m15949075919@163.com \\14130140356
    \and 戚瑶 \\ 631987611@qq.com \\14130140362
    \and 乔新文 \\ starsriver@outlook.com \\14130140393  
    \and 殷熔磾 \\ yinrongdi@163.com \\ 14050120069 }
%%%%%%%%%%%%%%%%%%%%%%
% 数据结构大作业 报告导言区
%%%%%%%%%%%%%%%%%%%%%%


\setCJKmainfont{WenQuanYi Zen Hei}
% 使用 color 宏包
\makeatletter
\ifdefined\@NoPagckageColor %! 如果定义了 ,不包含宏包color
\relax
\else
\usepackage{color}
\fi
\makeatother



% 对链接等的处理
\makeatletter
\ifdefined\@NoPackageHyperref
\relax
\else\usepackage[colorlinks,linkcolor=blue,anchorcolor=blue,citecolor=red,bookmarksnumbered]{hyperref}
\fi
\makeatother


% 使用 代码环境的宏包
\makeatletter
\ifdefined\@NoPackageListings
\relax
\else
\usepackage{listings}
\usepackage{xcolor}
\lstset{breaklines}
\lstset{basicstyle=\sffamily,keywordstyle=\bfseries,commentstyle=\rmfamily\itshape,escapechar='}
\lstset{flexiblecolumns}
\fi
\makeatother


%调整样式
%修改页眉与页脚
\usepackage{fancyhdr}
\pagestyle{fancy}



%章节的
\usepackage{titlesec}
%目录的
\usepackage{titletoc}



%修改Chapter的格式
\makeatletter
\ifdefined\@NoStyleChapter
\relax
\else\titleformat{\chapter}
    [display]
    {\centering\Huge\bfseries}
    {第\,\thechapter\,部分}
    {1em}
    {}
\fi
%修改Chapter的目录格式
\titlecontents{chapter}
[0pt]
{\addvspace{2pt}\filright}
{\contentspush{}}%\thecontentslabel\ }}
{a}
{\titlerule*[8pt]{.}\contentspage}
\makeatother

%定义 BibTeX 图标
\def\BibTeX{{\rm B\kern-.05em{\sc i\kern-.025em b}\kern-.08em
		T\kern-.1667em\lower.7ex\hbox{E}\kern-.125emX}}


%对于 api 版本的控制
\makeatletter
\ifx \apiver \@empty
\def\apiver{!!}
\fi
\ifdefined\@APIVersionFlag
\def\Color@now#1{\colorbox[rgb]{0.98,0.72,0.43}{#1}}
\def\Color@old#1{\colorbox[rgb]{0.79,0.79,0.58}{#1}}
\def\APIColor#1#2{
    \def\tmp@a{#1}
    \def\tmp@b{#2}
    \ifx \tmp@a \tmp@b
    \Color@new{#1}
    \else
    \Color@old{#1}
    \fi
    }
\def\apiversionn#1{\paragraph{\APIColor{\small#1}{\apiver}}}
\def\apiversion#1#2{\paragraph{{\colorbox[rgb]{0.98,0.72,0.43}{#1}} #2}\stepcounter{paragraph}}
\else
\def\apiversion#1{#1}
\def\apiversionn#1{#1}
\fi
\makeatother




%对于附录的设置
\makeatletter
\ifdefined\@UsingAppendix
\usepackage[titletoc]{appendix}
\renewcommand\appendixname{附录}
\renewcommand\appendixpagename{附录}
\appendixtitleon
\else
\relax
\fi
\makeatother

% 设置HTTP请求参数显示的背景色
\makeatletter
\ifdefined\@RequestArgColor
\def\reqargclr#1{\item[{\colorbox[rgb]{1.00,0.90,1.00}{#1}}]}
\fi
\makeatother


% 使得脚注标号随页码更新
\makeatletter

\ifdefined\@FootnoteWithPage
\usepackage{chngcntr}
\counterwithin{paragraph}{section}
\counterwithin{paragraph}{subsection}
\counterwithin{paragraph}{subsubsection}
\counterwithin{footnote}{paragraph}
\renewcommand\thefootnote{\theparagraph-\alph{footnote}}
\fi
\makeatother


% 定义 JSON
\makeatletter
\ifdefined\@NoPackageListings
\relax
\else
\ifdefined\@lst@json@define
\lstdefinelanguage{JSON}
{
    keywords={true,false,null}
}
\fi
\lstdefinelanguage{JavaScript}%
{
    morekeywords={true,false,null},%
    alsoletter={:},
    moredelim=[s][{\color[rgb]{0.67,0.00,0.67}}]{"}{"},
    moredelim=[s][{\color[rgb]{0.36,0.67,0.78}}]{:"}{"},%
    identifierstyle=\color{blue},
    emph={:}, emphstyle=\color{red}
}[keywords,comments,strings]%
\fi
\makeatother


\title{Yrarbil's API Document \thanks{version : \apiver}}

%%%%
%% API 常用
%%%%
\def\safe{/[\textit{safe}]}
\def\GET{\colorbox[rgb]{0.77,0.53,0.97}{\textbf{GET}}}
\def\POST{\colorbox[rgb]{0.77,0.53,0.97}{\textbf{POST}}}
\def\bfJSON{\textbf{JSON}}
\def\viaurl{\item[{\quad\colorbox[rgb]{0.47,0.88,0.89}{访问URL}}]}
\def\viareq#1{\item[{\quad\colorbox[rgb]{0.57,0.88,0.99}{访问方式}}] #1}
\def\rtdata{\item[{\quad\colorbox[rgb]{0.70,0.9,0.59}{返回数据}}]}





\begin{document}
    \maketitle
    \newpage
    
%%%%%%%%%%%%%%%%%%%%%%%%%
% 数据结构大作业 项目介绍
%%%%%%%%%%%%%%%%%%%%%%%%%

\section*{项目介绍}
\subsection*{团队介绍}
XDUDsTeam 开发团队由 西安电子科技大学软件学院1413014班的五名同学组成。他们依次是:
\paragraph{李约瀚} 李约瀚就读于 西安电子科技大学软件学院软件工程专业,目前是一名大二学生。
主要使用 Haskell C\# C++ 等语言进行开发。
\subsection*{项目介绍}
该项目是一个简单地图书馆图书管理系统。可用于小型图书馆的图书管理工作。

该项目是起于西安电子科技大学软件学院2015年数据结构课程大作业。目前遵循 BSD3 开原协议 \footnote{有计划更换开原协议。}。这个项目后端使用以 Haskell 编写的框架 wrap 与 wai \footnote{其中主要是用到了一个名为 \href{https://www.yesodweb.com}{Yesod} 的框架}。同时使用著名的开源数据库
\href{http://www.postgresql.org}{PostgreSQL} ,作为存储数据的方式。
\subsection*{项目信息}
这个项目是遵行 BSD3 开源协议开源的。托管于 GitHub 上的   \href{https://github.com/XDUDsTeam/}{Repo}
    \pdfbookmark[1]{目录}{anchor}
    \tableofcontents
    \newpage


    \section{API 改动说明}

    \subsection{API 0.0.0.0-dbtag}
    该 API 标志的,首先,意味着这个数据库可查询其中支持的各个 API 版本。其次,意味着这是一个标准
      \footnote{按照这个API 文档构件的后端。并且从理论上,非配套的前后端也可以交互。}
    后端。最后,一般前端是无法直接访问数据库的,然而后端是要访问数据库的。

    \subsection{API 0.0.0.0-betag}
    该 API 标志着可以通过特定方式查询后端与数据库支持的 API 的版本。这也是一个标准后端的应有的。

    \subsection{DEL 0.0.0.0-tag,0.0.0.0-B,0.0.1.0-base}
    重新 整理API。

    \section{API说明}
    本 API 文档是约束前后端的行为的文档。规定了前后端之间的行为规范。但值得注意的是,前后端不一定是满足这个文档的。当前的 API 版本是 \apiver 。
    \subsection{那些部分会遵循api}
    原则上来说,整个项目均会按照此 api 文档构建。但主要的是在数据库,后端,与前端部分支持。













    \section{数据库API}
    \apiversion{0.0.0.0-dstag}{版本检查}
    可查询表 versions 来确定 所需要的 api 对应的版本号是否支持。
    \begin{lstlisting}[language=SQL]
SELECT * FROM versions WHERE NOT vereq(ver,yourver);
    \end{lstlisting}
    其中 yourver 是指定的版本标号。




















    \section{后端API}
    后端 API 主要以 HTTP 请求为主。

%%%%%%%%%%%%%%%%%%%%%%%%%%%%%%%%%%%%%%%%%%%%%%%%%%%%%%%%%%%%%%%%%%%%%%%%%%%%%%%%%%%%%%%%%%%%%

    \subsection{未分类API}

    \apiversion{0.1.0.0-base}{访问“主页”}
    访问主页
    \footnote{ http://hostname:port/ hostname 是指 当前后端运行的主机的ip,或者其他URL 等可以访问的域名,ip等有效内容,port是端口,默认是3000}
    可以获取一些信息。
    \begin{description}
        \viaurl http://hostname:port/
        \viareq{\GET}
        \rtdata 标准 HTML 代码。
    \end{description}

    \apiversion{0.0.1.0-base}{获得 api 版本}
    获得当前 api 版本支持。
    \begin{description}
		\viaurl http://hostname:port/version
		\viareq{\GET}
		\reqargclr{NoJSON 选项}
		$$nojson=[true|false]$$
		如果是 $true$ 那么则不返回 JSON 数据,默认是 $false$。
        \rtdata 返回数据时 \bfJSON 数据,其中 将会返回数据库与后端的版本。
	\end{description}

%%%%%%%%%%%%%%%%%%%%%%%%%%%%%%%%%%%%%%%%%%%%%%%%%%%%%%%%%%%%%%%%%%%%%%%%%%%%%%%%%%%%%%%%%%%%%


    \subsection{用户认证API}%   Authentication



    \apiversion{0.1.0.0-base}{管理员登陆}
    \begin{description}

        \viaurl http://hostname:port/1daa62b
        \footnote{
            英文单词 \textbf{authentication} 的
            MD5 值是
            $1daa54e7c5a9caaa0c5043eadea9d824$,
            SHA256 值是
            $557212052a5022b1055b5c259eadd48a13c79a360215300c5753d17870d4a62b$。
        }
        \safe/b8ab91a6
        \footnote{
            英文单词\textbf{login}  MD5 值是$b8abe8f5a587a4c69685b17a954f8a3f$,
            SHA256值是
            $6c1420b8f3be7a61357ecc82a1c391dffe438930c103466b1660625e42ac91a6$。
        }
        \safe
        \footnote{\textit{\textbf{safe}} 是指安全添加的内容,例如可以是主机ip的sha256值。一个URL中的“safe”不一定相同。后同}/fc8252b7
        \footnote{
            英文单词的\textbf{admin}
            MD5值是 $456b7016a916a4b178dd72b947c152b7 $,
            SHA256 值是
            $fc8252c8dc55839967c58b9ad755a59b61b67c13227ddae4bd3f78a38bf394f7$。
        }
        \safe

        \viareq \POST

        \reqargclr{用户ID}\label{par:id}
        \footnote{这个颜色的背景,表示HTTP请求参数,后同。}
        $$uid=[id]$$
        其中id为用户登录的id,不强制为数字。

        \reqargclr{密码}\label{par:key}
         $$key=[md5\, of\, password][safe]$$
         其中第一部分是密码的md5值。

        \rtdata 请求返回的数据是 \bfJSON 数据。具体返回内容有:
        特定的临时ID密钥,用于简化之后操作的认证;
        时限,长度为小时,限定使用实现\footnote{可选,可无}。
         \\ 可能的返回数据:
        \begin{lstlisting}[language=JavaScript]
{
   "tidk":"************************",
   "time":24
}
        \end{lstlisting}


    \end{description}








    \apiversion{0.1.0.0-base}{管理员退出}
    \begin{description}
        \viaurl http://hostname:port/1daa62b\safe/a050ba73
        \footnote{
            英文单词 \textbf{logout} 的
            MD5值是$a05080965b04f4d190bb94b552bea56c $,
            SHA256 值是
            $60b2a0ba73317dce9502417ce5bdb75d0a16067848647b9a5f5f4e040397b20f$。
        }
        \safe/fc8252b7\safe

        \viareq \POST

        \reqargclr{用户ID}\label{par:logout:id}
        \footnote{这个请求与临时ID密钥为二选一。}
        $$uid=[id]$$
        其中id为用户登陆时的id,不强制为数字。

        \reqargclr{临时ID密钥}\label{par:logout:tidk}
        \footnote{用户ID,可使该用户对应的所有临时ID密钥下线。}
        $$tidk=[key]$$
        其中key是临时ID密钥。

        \rtdata 返回数据将是 \bfJSON 数据。其中会包含成功与否的信息。
        \\可能返回的信息:
        \begin{lstlisting}[language=JavaScript]
{
    "status":"success",
    "logoutcount":2
}
        \end{lstlisting}\label{par:logout:example}
        其中 \textbf{logoutcount} 是退出用户数量。
    \end{description}






    \apiversion{0.1.0.0-base}{读者登录}
		用于读者的登录。
		\begin{description}

		\viaurl http://hostname:port/1daa62b\safe /b8ab91a6\safe/5ece85ba
        \footnote{
            英文单词 \textbf{reader} 的
            MD5 值是
            $5ece7ae3f6950b744840b1d4c42e3a8e$,
            SHA256 值是
            $cd546fe85ba55d958a7fd1b2733f6f8f1ba5e4e099ddf0aad80b6d770b2eed9e$。
            }
        \safe

        \viareq \POST

        \reqargclr{用户ID} 参见 \ref{par:id}。
        \reqargclr{密码} 参见 \ref{par:key}。
        \rtdata 返回的数据将是 \bfJSON 数据。返回的内容中包括临时ID密钥,与有效时间,以分钟计。
        \\可能返回的数据:
        \begin{lstlisting}[language=JavaScript]
{
    "tidk":"****************",
    "time":120
}
        \end{lstlisting}
		\end{description}





        \apiversion{0.1.0.0-base}{读者退出}
        \begin{description}
            \viaurl http://hostname:port/1daa62b\safe/5ece85ba\safe/fc8252b7
            \viareq \POST

            \reqargclr{用户ID}
            参见 \ref{par:logout:id}。

            \reqargclr{临时ID密钥}
            参见 \ref{par:logout:tidk}。


            \rtdata 返回数据将是 \bfJSON 数据。其中会包含成功与否的信息。
            \\可能返回的信息:
            \begin{lstlisting}[language=JavaScript]
{
    "status":"failure"
}
            \end{lstlisting}
            另外一个例子请参见 \ref{par:logout:example}。
        \end{description}


        \subsection{图书借阅API}
        \footnote{借阅,归还,续借}
        \subsubsection*{注意}
        这一部分需要管理员的临时ID 密钥,其给定的参数是
        $$tid=[temp\,id\,key]$$。
        其中 temp id key 是临时秘钥,其中要求该部分中的每个 POST GET 访问 提供。
        \apiversion{0.1.0.0-base}{图书借阅}
        \begin{description}
            \viaurl http://hostname:port/a3cab3a
                \footnote{
                    英文单词 \textbf{management} 的
                    MD5值是$759d05d9478213e50a48a3ca7545fdcd$,
                    SHA256 值是
                    $9b47b1d1f65f93872d2b1cc222ab3a5b842c9213c0601b6fab6bf0b274b6c823$。
                }
                \safe/5527da67
                \footnote{
                    英文单词 \textbf{lending} 的
                    MD5 值是
                    $07c52effd9cc6b16d231c7da672826a4$,
                    SHA256 值是
                    $e0bdc575f0fa8f791e99d552763f6a57353b940be7375a2bdba2e034387daa4e$。
                }
                \safe

                \viareq{\POST}

                \reqargclr{读者条码}
                $$rid=[reader's\,id]$$
                给定读者的条码。
                \reqargclr{图书条码}
                $$bid=[book‘s\,id]$$
                给定图书的条码。
                \rtdata
                返回的数据是 \bfJSON 数据。会显示借阅成功与否\footnote{如果失败,将会传入失败的原因。超期可能使得图书无法借阅。},何时到期。
                \\  可能返回的数据
                \begin{lstlisting}[language=JavaScript]
{
	"status":"success",
	"date":"2010-2-29"
}
                \end{lstlisting}
        \end{description}
        \apiversion{0.1.0.0-base}{图书归还}
        \begin{description}
	        \viaurl  http://hostname:port/a3cab3a\safe/3689884
	        \footnote{
	        	英文单词 \textbf{return}的
	        	MD5 值是 $64672a10de469fbe6988486f9582d37c$,
	        	SHA256值是$9e34e5324d74f5d7636894144e2b8804df7ae09ca2268fc207a328e4811b8200$。
	        }\safe
	        \viareq{\POST}
	        \reqargclr{图书条码}
	        $$bid=[book's\,id]$$
	        给定要归还的图书的目录。
	        \rtdata 返回数据为\bfJSON 。或者还书成功,或者返回失败的信息及原因。如果图书超期,则需要先处理超期
	        \footnote{即缴纳罚款,以“购买”借阅时长,使得可以还款,此处需要先除法超期处理,然后再缴纳罚款,最后还书。}。
	        \\ 可能返回的数据
	        \begin{lstlisting}[language=JavaScript]
{
	"status":"failed",
	"reason":"overdue"
}
	        \end{lstlisting}
	        \end{description}
        \apiversion{0.1.0.0-base}{图书续借}
        \begin{description}
	        \viaurl http://hostname:port/a3cab3a\safe/4a2a356d
		     \footnote{
		     	英文单词 \textbf{renew} 的
		     	MD5 值是$faa1a11b00a356df6bf38236a8bf8437$,
		     	SHA256 值是 $4a2ae1cd00fe13c16af14299ef08f8aedd0722832723a1e1f683eb1308950b0d$。
		     }\safe
		     \viareq{\POST}
		     \reqargclr{图书条码}
		     $$bid=[book's\,id]$$
		     给定要归还的图书的条码。
		     \rtdata 返回的数据为 \bfJSON 。或者成功,并返回借阅期限。或者返回的是原因,如果为超期,则先需要处理超期问题。
		     \\ 可能返回的数据:
		     \begin{lstlisting}[language=JavaScript]
{
	"status":"success",
    "returnDate":"2020-11-31"
}
		     \end{lstlisting}
\footnote{2020-11-31只是一个例子,实际不会返回这类数据。}
		      \end{description}
        \apiversion{0.1.0.0-base}{超期处理}
        此处,超期处理的核心本质是通过购买借阅时长,使得借阅,还书,续借得以正常
        \footnote{如果有图书超期,只有归还其他未超期图书与购买图书期限不会受到限制}。
        \begin{description}
            \viaurl http://hostname:port/a3cab3a\safe/485e205c
            \footnote{
               英文单词 \textbf{buy-time} 的MD5 值是
                $480cd611d3ca7d78abe08436fb05d95c$,
                SHA256 值是
                $5e580e1956693597ba206e4bf82f24b59fef00e3f4a4ce19ab4bb13ed67d8520$。
            }\safe
            \viareq{\POST}
            \reqargclr{大流水号}
            $$bsm=[the big\,serial\,number ]$$
            给定大流水号。
            \rtdata 返回的数据将会有处理成功与否,和延长的时间。
            \\ 可能返回的数据
            \begin{lstlisting}[language=JavaScript]
{
    "status":"success",
    "add":10
}
            \end{lstlisting}
        \end{description}
        \apiversion{0.1.0.0-base}{处罚处理}%缴纳罚款
        \begin{description}
	        \viaurl http://hostname:port/a3cab3a\safe/c91ed8fd
	        \footnote{英文单词 \textbf{pay} 的
	        	MD5 值是 $c91e7f018a4ea68d6864a7d21f663c9a$,
	        	SHA256 值是 $4f675fed8fd4763a87b0981acb0cf446b1a438233aef624f563844b97d5c7bf5$。
	        	}
	        \safe
	        \viareq{\POST}
	        \reqargclr{读者条码}
	        $$rid=[reader's\,id]$$
	        传入的是读者条码。
	        \reqargclr{金额}
	        $$cash=[reader's\,cash]$$
	        传入的是读者的缴纳的罚款的金额。
	        \reqargclr{事由}
	        $$resaon=[text]$$
	        传入的是为什么缴纳罚款。
	        \rtdata 返回值是 \bfJSON 数据,其中包含是否成功,如果不成功会给出理由,反之则会给出处理的大流水号。
	        \\ 可呢的例子:
	        \begin{lstlisting}[language=JavaScript]
{
	"status":"success",
	"bigSerialNumber":"4654231@PAY@2015-10-24"
}
	        \end{lstlisting}
	    \end{description}
        \subsection{图书预约API}
        这一部分需要给定读者的临时 ID 秘钥,格式是
        $$tidk=[the\,temp\,id\,key\,of\,reader]$$
        其中读者秘钥 需要提供给 所有 POST GET 方法。
        \apiversion{0.1.0.0-base}{图书预约}
        \begin{description}
	         \viaurl http://hostname:port/1d224726
	         \footnote{
	         	英文单词 \textbf{appointment} 的
	         	MD5 值是 $1d226c3a8e1e3eb2cdf2d4fba102f020$,
	         	SHA256值是 $4b247264036a8c154cd24bd40987cae6bdf7d3d9793b2d81dbe39571cc64a10a$。
	         }\safe/on
		     \viareq{\POST}
		     \reqargclr{图书条码}
		     \rtdata 返回的数据是 \bfJSON ,将返回是否预约成功,以及如果预约不成功将返回预约失败的原因。
		     \\ 肯能返回的内容:
			 \begin{lstlisting}[language=JavaScript]
{
	"status":"failed",
	"reason":"Is not in library"
}
			 \end{lstlisting}
	       \end{description}
        \apiversion{0.1.0.0-base}{图书预约取消}
        \begin{description}
	        \viaurl http://hostname:port/1d224726\safe/off
	        \viareq{\POST}
	        \reqargclr{图书条码}
            $$barcode=[the\, barcode\,of\,book]$$
            给入图书的条码。
	        \rtdata 返回的数据是 \bfJSON ,其中包括 是否成功,以及不成功的原因。
            \\ 可能返回的数据:
            \begin{lstlisting}[language=JavaScript]
{
    "status":"success"
}
            \end{lstlisting}
	         \end{description}
        \subsection{常量查询}
        \apiversion{0.1.0.0-base}{公开常量查询}
        \begin{description}
	        \viaurl http://hostname:port/varcat/public
	        \viareq{\POST}
	        \reqargclr{查询的变量名}
	        $$varname=[text]$$
	        \rtdata 返回的数据是 \bfJSON ,内容包括所查询的内容与结果。
	        \\ 可能的返回:
			\begin{lstlisting}[language=JavaScript]
{
	"requested":"undefined",
	"retrun":"It just undefined :)"
}
			\end{lstlisting}
			 \end{description}
        \subsection{图书信息查询}
        \apiversion{0.1.0.0-base}{图书状态查询}
        \begin{description}
	        \viaurl http://hostname:port/423472651c\safe/ons
	        \viareq{\GET}
	        \reqargclr{图书条码}
	        $$bid=[book's\,id]$$
	        或
	        $$isbn=[book's\,isbn]$$
	        给定图书的条码或 ISBN 码。
	        \rtdata 返回的数 JSON 数据,其中
	        \begin{lstlisting}[language=JavaScript]
{
	"state":"on"
}
	        \end{lstlisting}
	         \end{description}
        \apiversion{0.1.0.0-base}{图书CIP信息查询}
                通过 ISBN查询,返回图书的信息。
                \begin{description}

                    \viaurl http://hostname:port/423472651c
                    \footnote{单词 \textbf{book}的
                        MD5 值是 $4f98663772651c870e911982e991d0d9$,
                        SHA256 值是
                        $3c4dbff9eeda9410a0b9eb423472981db8997d666388b4cd2424700c3974d64b$
                        。
                    }
                    \safe/9c4f7bfef
                    \footnote{
                        单词 \textbf{search} 的
                        MD5 值是 $9c4fc45b8e6ad4bf08f959548f52099e$,
                        SHA256 值是
                        $f1f3528bad895d7e9db4544081b531c2126781843aebafa6cbea7c2f9358cf85$
                        。
                        单词 \textbf{ISBN} 的
                        MD5 值是 $f329188d04f7bd852894eed846386423$,
                        SHA256 值是
                        $3cb720e9f89bfef38a513b1ee0c47cc0ae75c3353612c3466761df19dc945f29$
                        。
                    }
                    \safe


                    \viareq \GET


                    \reqargclr{ISBN码}
                    $$isbn=[ISBN]$$
                    其中的 ISBN 是书的 ISBN 码。


                    \reqargclr{返回数据形式}
                    $$type=[json|html]$$
                    返回的数据格式类型是 JSON 数据还是 HTML 文件。

                    \rtdata 请求返回的可能是数据,也可能是 html。
                    可能返回的数据之一:
                    \begin{lstlisting}[language=JavaScript]
{
    "name":"the name of book",
    "author":"who write it!",
    "publishLocation":"Xi`an,China",
    "pressHouse":"ourPress",
    "publishDate":"1970-01-01",
    "ISBN":"ISBN978-7-302-22446-4",
    "location":"TP312"
}
                    \end{lstlisting}
                    或者会拿到 HTMl 代码:
                    \begin{lstlisting}[language=HTML]
<HTML>
    <body>
        <p> the name of book</p>
        <p> 'who write it' write this book </p>
        <p> 'our press' publish this book in Xi`an , China </p>
        <p> At Jan. 1st, 1970. </p>
        <p> ISBN is 978-7-302-22446-4</p>
        <p> You can find these books at TP312.</p>
    </body>
</HTML>
                    \end{lstlisting}
                \end{description}
        \apiversion{0.1.0.0-base}{图书其他信息查询}
        暂时先不做其他的数据检索。
        \subsection{图书检索}
        \apiversion{0.1.0.0-base}{根据ISBN匹配}
        \apiversion{0.1.0.0-base}{根据条件检索}
        \apiversion{0.1.0.0-plus}{获取图书的\BibTeX 引用数据}
        \subsection{图书录入与销毁}
        \apiversion{0.1.0.0-base}{图书录入}
        实体的单本图书的录入。
        \begin{description}
            \viaurl  http://hostname:port/c7b70ceb
            \footnote{
                英文单词\textbf{bookchange}的
                MD5 值是 $c7b74167f5663260ce68b76a8636407a $,
                SHA256 值是 $0cebc922561ea3f51cff44858e854ff1aee6454d245e85a23bb42e3e9b5cd365$。
            }
            \safe
            /8769ab8
            \footnote{
                英文单词 \textbf{add} 的
                MD5 值是 $69a904ac9adfbefaa69a720e4c13780a $,
                SHA256 值是 $87623ceab830648b4f91539f54b7d3d6ef9bbdb04a1442dcfbf90103dd60e73e$。
            }\safe

            \viareq{\POST}

            \reqargclr{ISBN} 输入ISBN 数据。
            $$isbn=[the\,isbn\,of\,book]$$
            \reqargclr{条码} 获得条码。
            $$barcode=[the\,barcode\,of\,book]]$$
            \rtdata 返回的数据是 \bfJSON 数据,将返回是否成功,如果失败,将返回原因。
            \\ 可能的返回数据:
            \begin{lstlisting}[language=JavaScript]
{
    "status":"success"
}
            \end{lstlisting}
        \end{description}
        \apiversion{0.1.0.0-base}{图书录入}
        图书录入,并非实体。
        \begin{description}
        \viaurl http://hostname:port/c7b70ceb\safe/4d632ab9
        \footnote{
            英文单词 \textbf{delete} 的
            MD5 值是 $4d637d37b5ef46274393899205774420$,
            SHA256 值是 $2ab99e684919afa1245556d366a2018d1dd0bd63b1b66fb92cde34ec5c4d60c6$。
        }\safe
        \viareq{\POST}
        \reqargclr{ISBN}
        $$isbn=[the\,isbn\,of\,book]$$
        提供图书的ISBN码。
        \reqargclr{书名}
        $$title=[the\,title\,of\,book]$$
        提供图书的书名/题目。
        \reqargclr{作者}
        $$auth=[the\,author\,of\,book]$$
        提供作者是谁。
        \reqargclr{出版地}
        $$publocal=[the\,localtion\,of\,publish]$$
        提供图书出版地的信息。
        \reqargclr{出版社}
        $$pubh=[the\,publish\,house\,of\,book]$$
        提供图书的出版社信息。
        \reqargclr{出版日期}
        $$pubd=[the\,date\,of\,publish]$$
        提供图书出版的时间。
        \reqargclr{图书分类} 获得图书的分类信息

        $$zth=[zth]$$
        提供图书的中国图书分类号。
        \rtdata 返回提供的是否是成功。
	    	\\ 可能返回的数据:
        \begin{lstlisting}[language=JavaScript]
{
    "status":"success"
}
		\end{lstlisting}
        \end{description}
        \apiversion{0.1.0.0-base}{图书销毁}
        \begin{description}
		\viaurl http://hostname:port/c7b70ceb\safe/761b31a3
        \footnote{
            英文单词 \textbf{destroy} 的MD5 值是
            $7615bbace0fdc952fa5b4067f18a2d7a$,
            SHA256 值是
            $b3188b89408c7d8a17f3a985bd087051ce21c5cfe56ce8f9962b79f3c793c8d3$。
        }\safe
        \viareq{\POST}
        \reqargclr{图书条码}
        				$$barcode=[the\,barcode\,of\,the\,book]$$
          给入 说要删除的特定一本图书的条码。
        \reqargclr{ISBN码}\footnote{这个参数与图书条码为至少二选一。}
          $$isbn=[the\,books'\,isbn\,code]$$
          给入所要删除的书的ISBN,然后把所有ISBN码符合标准的图书注销。
        \rtdata 返回的数据是 \bfJSON 数据。
        \\ 可能返回的数据是:
        \begin{lstlisting}[language=JavaScript]
{
    "status":"failed",
    "reason":"no such a book"
}
        \end{lstlisting}
        \end{description}
		\apiversion{0.1.0.0-plus}{获取下一个可用的图书的条码}
        \apiversion{0.1.0.0-plus}{查询某一个条码是否对应图书}
        \subsection{数据分析接口}
        \apiversion{0.1.0.0-da}{获得所有图书的ISBN码}
        \apiversion{0.1.0.0-da}{通过ISBN码查找对应的图书的条码}
        \apiversion{0.1.0.0-da}{通过ISBN码查询某一个图书的借阅次数}

		\section{错误代码}
% 借阅期限,续借次数,超期罚款

		\section{可查询的变量名称}



    \newpage
    \begin{appendices}
    \section{参考文档}
    \begin{enumerate}
        \item Yrarbil 的需求分析报告
        \item Yrarbil 的架构分析报告
        \item Yrarbil 的测试报告
        \item Yrarbil 的后端文档
    \end{enumerate}
    \section{文档网址}
         全部的文档可以访问 \url{https://github.com/XDUDsTeam/YrarbilRelease} 仓库。
      目前的所有的文档,均通过 Travis-ci
      \footnote{Travis CI是在软件开发领域中的一个在线的,分布式的持续集成服务。}
      及时编译,再次并上传到GitHub上。

    \end{appendices}

\end{document}
