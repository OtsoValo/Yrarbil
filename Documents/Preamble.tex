%%%%%%%%%%%%%%%%%%%%%%
% 数据结构大作业 报告导言区
%%%%%%%%%%%%%%%%%%%%%%

% 使用 color 宏包
\makeatletter
\ifdefined\@NoPagckageColor %! 如果定义了 ,不包含宏包color
\relax
\else\usepackage{color}
\fi
\makeatother



% 对链接等的处理
\makeatletter
\ifdefined\@NoPackageHyperref
\relax
\else\usepackage[colorlinks,linkcolor=blue,anchorcolor=blue,citecolor=red,bookmarksnumbered]{hyperref}
\fi
\makeatother


% 使用 代码环境的宏包
\makeatletter
\ifdefined\@NoPackageListings
\relax
\else\usepackage{listings}
\fi


%调整样式
%修改页眉与页脚
\usepackage{fancyhdr}
\pagestyle{fancy}



%章节的
\usepackage{titlesec}
%目录的
\usepackage{titletoc}



%修改Chapter的格式
\makeatletter
\ifdefined\@NoStyleChapter
\relax
\else\titleformat{\chapter}
    [display]
    {\centering\Huge\bfseries}
    {第\,\thechapter\,部分}
    {1em}
    {}
%修改Chapter的目录格式
\titlecontents{chapter}
    [0pt]
    {\addvspace{2pt}\filright}
    {\contentspush{}}%\thecontentslabel\ }}
    {a}
    {\titlerule*[8pt]{.}\contentspage}
\fi
\makeatother

%定义 BibTeX 图标
\def\BibTeX{{\rm B\kern-.05em{\sc i\kern-.025em b}\kern-.08em
		T\kern-.1667em\lower.7ex\hbox{E}\kern-.125emX}}


%对于 api 版本的控制 
\makeatletter
\ifdefined\@APIVersionFlag
\def\apiversionn#1{\paragraph{\colorbox[rgb]{0.98,0.72,0.43}{\small#1}}}
\def\apiversion#1{\paragraph*{\colorbox[rgb]{0.98,0.72,0.43}{\small#1}}}
\else
\def\apiversion#1{\relax}
\def\apiversionn#1{\relax}
\fi
\makeatother

